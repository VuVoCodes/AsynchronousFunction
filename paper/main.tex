\documentclass[11pt]{article}
\usepackage{neurips_2026}

% Additional packages
\usepackage{amsmath,amssymb,amsfonts}
\usepackage{graphicx}
\usepackage{booktabs}
\usepackage{algorithm}
\usepackage{algorithmic}
\usepackage{multirow}
\usepackage{subcaption}
\usepackage{xcolor}
\usepackage{bm}

% Custom commands
\newcommand{\method}{ASGML}
\newcommand{\methodfull}{Asynchronous Staleness Guided Multimodal Learning}
\newcommand{\R}{\mathbb{R}}
\newcommand{\E}{\mathbb{E}}
\newcommand{\indicator}{\mathbbm{1}}

\title{\methodfull}

% Anonymous submission
\author{
  Anonymous Author(s)\\
  \texttt{anonymous@email.com}
}

\begin{document}

\maketitle

\begin{abstract}
Multimodal learning often suffers from the modality imbalance problem, where dominant modalities suppress the learning of weaker ones during training. Existing approaches address this through gradient modulation or loss reweighting, but all maintain synchronous updates across modalities. We argue that true asynchronous optimization—where modalities update at different frequencies—remains unexplored and offers distinct theoretical and practical advantages. We propose \methodfull{} (\method{}), a novel approach that introduces adaptive staleness control inspired by distributed systems theory. \method{} monitors each modality's learning dynamics through gradient magnitude ratios and loss descent rates, then adjusts update frequencies to prolong the \textit{Prime Learning Window}—the critical early training phase where all modalities can learn effectively. By throttling fast-learning modalities and accelerating slow ones through genuine update skipping (not just gradient scaling), \method{} maintains balanced learning throughout training. Experiments on CREMA-D, AVE, and Kinetics-Sounds demonstrate that \method{} outperforms state-of-the-art methods including OGM-GE, PMR, MMPareto, and ARL, achieving improvements of X\% on average across benchmarks.
\end{abstract}

%------------------------------------------------------------------------------
\section{Introduction}
\label{sec:intro}
% Introduction

Multimodal learning has achieved remarkable success across diverse applications including action recognition~\citep{}, audio-visual speech recognition~\citep{}, and sentiment analysis~\citep{}. By leveraging complementary information from multiple modalities, multimodal models have the potential to significantly outperform their unimodal counterparts. However, realizing this potential remains challenging due to the \textit{modality imbalance problem}—a phenomenon where dominant modalities suppress the learning of weaker ones during joint training~\citep{wang2020makes,peng2022balanced}.

Recent work has identified the \textit{Prime Learning Window} as a critical factor in multimodal optimization~\citep{}. During early training, gradient dynamics can cause networks to settle into suboptimal solutions that favor a single modality, effectively closing the window for other modalities to contribute meaningfully. Once this window closes, recovery becomes difficult, leading to under-optimized multimodal representations.

Existing approaches address modality imbalance through two main strategies. \textbf{Gradient modulation} methods~\citep{peng2022balanced,li2023boosting,fan2023pmr} dynamically adjust gradient magnitudes to balance learning across modalities. \textbf{Unimodal regularization} methods~\citep{wei2024mmpareto,zhang2023mla,wei2025diagnosing} introduce auxiliary losses to ensure each modality learns discriminative representations. Most recently, \citet{wei2025improving} proposed Asymmetric Representation Learning (ARL), arguing that \textit{imbalanced} learning based on modality variance ratios can outperform balanced approaches.

However, all existing methods share a fundamental limitation: they maintain \textbf{synchronous updates} across all modalities. Every training step updates every modality encoder, differing only in gradient magnitude or loss weighting. We argue that this overlooks a powerful mechanism from distributed systems: \textbf{asynchronous optimization with gradient staleness}.

In distributed SGD, workers can operate with \textit{stale gradients}—gradients computed on previous parameter states—enabling asynchronous updates that improve system efficiency without sacrificing convergence guarantees~\citep{}. The key insight is that staleness introduces a form of implicit regularization that can be carefully controlled through bounded-delay scheduling.

We propose to adapt this concept to multimodal learning: rather than just scaling gradients, we introduce \textbf{true asynchronous updates} where each modality has an adaptive update frequency. Fast-learning modalities receive higher staleness (fewer updates), while slow-learning modalities receive lower staleness (more frequent updates). This mechanism directly prolongs the Prime Learning Window by preventing any single modality from dominating the optimization trajectory.

Our contributions are as follows:
\begin{itemize}
    \item We identify a gap in multimodal learning: true asynchronous optimization with independent modality update schedules remains unexplored, despite its theoretical grounding in distributed systems.

    \item We propose \methodfull{} (\method{}), a novel loss function that introduces adaptive staleness control. \method{} monitors learning dynamics through gradient magnitude ratios and loss descent rates, then adjusts per-modality update frequencies accordingly.

    \item We provide theoretical analysis connecting \method{} to bounded-staleness SGD convergence guarantees, adapting these results to the multimodal setting.

    \item Extensive experiments on CREMA-D, AVE, and Kinetics-Sounds demonstrate that \method{} outperforms state-of-the-art methods, with particularly strong gains when modality imbalance is severe.
\end{itemize}


%------------------------------------------------------------------------------
\section{Related Work}
\label{sec:related}
% Related Work

\subsection{Multimodal Learning}

Multimodal learning combines information from multiple data sources to improve model performance and robustness~\citep{}. Fusion strategies range from early fusion (concatenating raw inputs) to late fusion (combining modality-specific predictions)~\citep{}. While multimodal models theoretically benefit from complementary information, practical training often yields suboptimal results where multimodal performance fails to exceed that of the best unimodal model~\citep{wang2020makes}.

\subsection{Modality Imbalance Problem}

\citet{wang2020makes} first identified that different modalities overfit and generalize at different rates, leading to suboptimal multimodal learning. This \textit{modality imbalance} problem has since received significant attention.

\textbf{Gradient Modulation.} OGM~\citep{peng2022balanced} reduces gradients of the dominant modality to mitigate suppression of weaker modalities. AGM~\citep{li2023boosting} extends this with adaptive modulation based on modality performance gaps. PMR~\citep{fan2023pmr} introduces prototypical rebalancing to accelerate slow-learning modalities.

\textbf{Unimodal Regularization.} G-Blending~\citep{wang2020makes} optimizes gradient mixing weights based on unimodal and multimodal performance. MMPareto~\citep{wei2024mmpareto} uses Pareto optimization to balance multimodal and unimodal objectives. MLA~\citep{zhang2023mla} transforms joint training into alternating unimodal learning.

\textbf{Imbalanced Learning.} Most recently, ARL~\citep{wei2025improving} challenges the assumption that balanced learning is optimal. Using bias-variance analysis, they show that optimal modality contribution should be inversely proportional to modality variance. However, ARL still updates all modalities synchronously.

\textbf{Our Distinction.} All existing methods maintain synchronous updates—every modality encoder receives gradient updates at every step, differing only in magnitude. \method{} introduces \textit{true asynchrony} with independent update schedules, a fundamentally different mechanism with distinct theoretical properties.

\subsection{Asynchronous Optimization}

Asynchronous SGD enables distributed training where workers compute gradients on potentially stale parameters~\citep{}. Convergence guarantees exist under bounded staleness assumptions~\citep{}. Recent work has explored adaptive staleness bounds~\citep{} and staleness-aware learning rate schedules~\citep{}.

While asynchronous optimization is well-studied in distributed systems, its application to \textit{modality-level} updates in multimodal learning is novel. We adapt staleness concepts from distributed SGD to control per-modality update frequencies, providing a new mechanism for balanced multimodal learning.


%------------------------------------------------------------------------------
\section{Method}
\label{sec:method}
% Method

We present \methodfull{} (\method{}), a novel approach for balanced multimodal learning through adaptive asynchronous updates. We first formalize the problem setting, then introduce our monitoring signals and staleness control mechanism.

\subsection{Problem Setting}

Consider a multimodal learning task with $N$ modalities $\{m_1, m_2, \ldots, m_N\}$. Let $\mathcal{D} = \{(\mathbf{x}_i^{m_1}, \mathbf{x}_i^{m_2}, \ldots, \mathbf{x}_i^{m_N}, y_i)\}_{i=1}^{|\mathcal{D}|}$ denote the dataset where $y \in \{1, 2, \ldots, C\}$ represents class labels.

Each modality $m_j$ has an encoder $\phi_j(\theta_j, \cdot)$ with parameters $\theta_j$, producing representations $\mathbf{z}_j = \phi_j(\theta_j, \mathbf{x}^{m_j})$. A fusion module $\psi(\theta_f, \cdot)$ combines modality representations to produce the final prediction:
\begin{equation}
    \mathbf{p}^f = \psi(\theta_f, \mathbf{z}_1, \mathbf{z}_2, \ldots, \mathbf{z}_N)
\end{equation}

In standard multimodal learning, all parameters $\{\theta_1, \ldots, \theta_N, \theta_f\}$ are updated at every training step. Our key insight is that modality encoders can benefit from \textit{different update frequencies} based on their learning dynamics.

\subsection{Monitoring Signals}

We track two complementary signals to assess each modality's learning state:

\textbf{Signal 1: Gradient Magnitude Ratio.} The gradient norm provides an instantaneous measure of learning intensity. For modality $m_i$ at step $t$:
\begin{equation}
    G_i(t) = \frac{\|\nabla_{\theta_i} \mathcal{L}_i\|}{\frac{1}{N} \sum_{j=1}^{N} \|\nabla_{\theta_j} \mathcal{L}_j\|}
\end{equation}
where $\mathcal{L}_i$ denotes the loss contribution from modality $i$. A ratio $G_i > 1$ indicates modality $i$ has larger-than-average gradients, suggesting it is dominating the optimization.

\textbf{Signal 2: Loss Descent Rate.} Gradient magnitude alone can be noisy. We complement it with a temporal signal tracking loss descent over a window of $k$ steps:
\begin{equation}
    D_i(t) = \frac{\Delta \mathcal{L}_i(t)}{\frac{1}{N} \sum_{j=1}^{N} \Delta \mathcal{L}_j(t)}
\end{equation}
where $\Delta \mathcal{L}_i(t) = \mathcal{L}_i(t-k) - \mathcal{L}_i(t)$ measures loss decrease. A ratio $D_i > 1$ indicates modality $i$ is learning faster than average.

\textbf{Combined Learning Speed Score.} We combine both signals:
\begin{equation}
    S_i(t) = \beta \cdot G_i(t) + (1 - \beta) \cdot D_i(t)
\end{equation}
where $\beta \in [0, 1]$ balances instantaneous and temporal information.

\subsection{Adaptive Staleness Control}

The learning speed score determines each modality's update frequency through an \textit{adaptive staleness threshold}:
\begin{equation}
    \tau_i(t) = \text{clamp}\left(\tau_{\text{base}} \cdot S_i(t), \tau_{\min}, \tau_{\max}\right)
\end{equation}
where:
\begin{itemize}
    \item $\tau_{\text{base}}$: baseline staleness (hyperparameter)
    \item $\tau_{\min} = 1$: minimum staleness (update every step)
    \item $\tau_{\max}$: maximum staleness bound (e.g., 5-10 steps)
\end{itemize}

\textbf{Update Decision.} At step $t$, modality $m_i$ updates its encoder parameters if and only if:
\begin{equation}
    \text{update}_i(t) = \mathbbm{1}\left[t \mod \lfloor\tau_i(t)\rfloor = 0\right]
\end{equation}

This introduces \textit{true asynchrony}: fast-learning modalities ($S_i > 1$) have higher staleness and update less frequently, while slow-learning modalities ($S_i < 1$) update more often.

\subsection{Gradient Compensation}

When a modality does update after $\tau_i$ steps of staleness, its gradient may be outdated. Following insights from distributed SGD~\citep{}, we apply staleness-aware gradient scaling:
\begin{equation}
    \mathbf{g}_i = \nabla_{\theta_i} \mathcal{L}_i \cdot (1 + \lambda \cdot \tau_i)
\end{equation}
where $\lambda$ controls compensation strength. This accounts for the ``missed'' gradient updates during the staleness period.

\subsection{Total Loss Function}

The complete \method{} loss function is:
\begin{equation}
    \mathcal{L}_{\text{ASGML}} = \mathcal{L}_{\text{CE}}(\mathbf{p}^f, y) + \gamma \sum_{i=1}^{N} u_i \cdot \mathbbm{1}[\text{update}_i]
\end{equation}
where:
\begin{itemize}
    \item $\mathcal{L}_{\text{CE}}(\mathbf{p}^f, y)$: cross-entropy loss on fused predictions
    \item $u_i = \mathcal{L}_{\text{CE}}(\mathbf{p}^{m_i}, y)$: unimodal regularization for modality $i$
    \item $\gamma$: regularization weight
    \item $\mathbbm{1}[\text{update}_i]$: indicator for whether modality $i$ updates
\end{itemize}

The unimodal regularization terms are only applied when the corresponding modality updates, ensuring computational efficiency.

\subsection{Training Algorithm}

Algorithm~\ref{alg:asgml} summarizes the \method{} training procedure.

\begin{algorithm}[t]
\caption{\method{} Training}
\label{alg:asgml}
\begin{algorithmic}[1]
\REQUIRE Dataset $\mathcal{D}$, iterations $T$, hyperparameters $\tau_{\text{base}}, \tau_{\max}, \beta, \lambda, \gamma$
\STATE Initialize loss history buffers for each modality
\FOR{$t = 1, \ldots, T$}
    \STATE Sample minibatch from $\mathcal{D}$
    \STATE Forward pass: compute $\mathbf{z}_1, \ldots, \mathbf{z}_N$ and $\mathbf{p}^f$
    \STATE Compute fusion loss $\mathcal{L}_{\text{CE}}(\mathbf{p}^f, y)$
    \FOR{each modality $m_i$}
        \STATE Compute $G_i(t)$ from gradient norms
        \STATE Compute $D_i(t)$ from loss history
        \STATE Compute $S_i(t) = \beta G_i(t) + (1-\beta) D_i(t)$
        \STATE Compute $\tau_i(t) = \text{clamp}(\tau_{\text{base}} \cdot S_i(t), 1, \tau_{\max})$
        \IF{$t \mod \lfloor\tau_i(t)\rfloor = 0$}
            \STATE Compute scaled gradient: $\mathbf{g}_i = \nabla_{\theta_i}\mathcal{L}_i \cdot (1 + \lambda\tau_i)$
            \STATE Update $\theta_i$ using $\mathbf{g}_i$
            \STATE Apply unimodal regularization $u_i$
        \ENDIF
    \ENDFOR
    \STATE Update fusion parameters $\theta_f$
\ENDFOR
\end{algorithmic}
\end{algorithm}

\subsection{Theoretical Connection to Bounded-Staleness SGD}

Our adaptive staleness mechanism connects to convergence results from distributed optimization. In bounded-staleness SGD, convergence is guaranteed when:
\begin{enumerate}
    \item Staleness is bounded: $\tau_i \leq \tau_{\max}$ for all $i$
    \item Learning rate is appropriately scaled with maximum staleness
\end{enumerate}

\method{} satisfies both conditions through the clamp operation and gradient compensation. The key extension is making staleness bounds \textit{adaptive per modality} based on learning dynamics rather than fixed system constraints. We provide detailed convergence analysis in Appendix~\ref{app:theory}.


%------------------------------------------------------------------------------
\section{Experiments}
\label{sec:experiments}
% Experiments

We evaluate \method{} on three audio-visual benchmarks and compare against state-of-the-art methods for balanced multimodal learning.

\subsection{Datasets}

\textbf{CREMA-D}~\citep{cao2014crema} is an audio-visual dataset for emotion recognition containing 7,442 video clips across 6 emotion categories. We use 6,698 samples for training and 744 for testing.

\textbf{AVE}~\citep{tian2018audio} contains 4,143 10-second videos for audio-visual event localization across 28 event classes. Following prior work~\citep{}, we extract frames from event-localized segments to create a classification dataset.

\textbf{Kinetics-Sounds (KS)}~\citep{arandjelovic2017look} focuses on 34 human action classes from Kinetics, comprising 19k video clips (15k train, 1.9k val, 1.9k test).

\subsection{Implementation Details}

Following~\citet{wei2025improving}, we use ResNet18 as the encoder backbone for all experiments. For visual input, we sample frames and resize to 224$\times$224. Audio is converted to spectrograms using librosa~\citep{}. We use late fusion via concatenation.

\textbf{Training.} Batch size 64, SGD optimizer with momentum 0.9, learning rate 1e-3, weight decay 1e-4, 100 epochs.

\textbf{\method{} Hyperparameters.} $\tau_{\text{base}}=2$, $\tau_{\max}=5$, $\beta=0.5$, $\lambda=0.1$, $\gamma=1.0$, loss window $k=10$.

\subsection{Baselines}

We compare against:
\begin{itemize}
    \item \textbf{Concatenation}: Vanilla late fusion baseline
    \item \textbf{Gradient modulation}: OGM-GE~\citep{peng2022balanced}, AGM~\citep{li2023boosting}, PMR~\citep{fan2023pmr}
    \item \textbf{Unimodal regularization}: G-Blending~\citep{wang2020makes}, MMPareto~\citep{wei2024mmpareto}, MLA~\citep{zhang2023mla}, D\&R~\citep{wei2025diagnosing}
    \item \textbf{Imbalanced learning}: ARL~\citep{wei2025improving}
\end{itemize}

\subsection{Main Results}

Table~\ref{tab:main} presents results on all three benchmarks. \method{} achieves state-of-the-art performance across all datasets.

\begin{table}[t]
\centering
\caption{Comparison with existing methods on CREMA-D, Kinetics-Sounds (KS), and AVE datasets. Bold indicates best, underline indicates second-best.}
\label{tab:main}
\begin{tabular}{lcccccc}
\toprule
\multirow{2}{*}{Method} & \multicolumn{2}{c}{CREMA-D} & \multicolumn{2}{c}{KS} & \multicolumn{2}{c}{AVE} \\
\cmidrule(lr){2-3} \cmidrule(lr){4-5} \cmidrule(lr){6-7}
& Acc & F1 & Acc & F1 & Acc & F1 \\
\midrule
Audio-only & 57.27 & 57.89 & 48.67 & 48.89 & 62.16 & 58.54 \\
Visual-only & 62.17 & 62.78 & 52.36 & 52.67 & 31.40 & 29.87 \\
\midrule
Concatenation & 58.83 & 59.43 & 64.97 & 65.21 & 66.15 & 62.46 \\
G-Blending & 68.81 & 69.34 & 67.31 & 67.68 & 67.40 & 63.87 \\
OGM-GE & 64.34 & 64.93 & 66.35 & 66.76 & 65.62 & 62.97 \\
AGM & 67.21 & 68.04 & 65.61 & 65.99 & 64.50 & 61.49 \\
PMR & 65.12 & 65.91 & 65.01 & 65.13 & 63.62 & 60.36 \\
MMPareto & 70.19 & 70.82 & 69.13 & 69.05 & 68.22 & 64.54 \\
MLA & 73.21 & 73.77 & 69.62 & 69.98 & 70.92 & 67.23 \\
D\&R & 73.52 & 73.96 & 69.10 & 69.36 & 69.62 & 64.93 \\
ARL & \underline{76.61} & \underline{77.14} & \underline{74.28} & \underline{74.03} & \underline{72.89} & \underline{68.04} \\
\midrule
\method{} (Ours) & \textbf{XX.XX} & \textbf{XX.XX} & \textbf{XX.XX} & \textbf{XX.XX} & \textbf{XX.XX} & \textbf{XX.XX} \\
\bottomrule
\end{tabular}
\end{table}

\subsection{Ablation Study}

We ablate the contribution of each component in \method{} on CREMA-D (Table~\ref{tab:ablation}).

\begin{table}[t]
\centering
\caption{Ablation study on CREMA-D. AS: Adaptive Staleness, GC: Gradient Compensation, UR: Unimodal Regularization.}
\label{tab:ablation}
\begin{tabular}{ccc|cc}
\toprule
AS & GC & UR & Acc & F1 \\
\midrule
& & & 58.83 & 59.43 \\
\checkmark & & & XX.XX & XX.XX \\
\checkmark & \checkmark & & XX.XX & XX.XX \\
\checkmark & & \checkmark & XX.XX & XX.XX \\
\checkmark & \checkmark & \checkmark & \textbf{XX.XX} & \textbf{XX.XX} \\
\bottomrule
\end{tabular}
\end{table}


%------------------------------------------------------------------------------
\section{Analysis}
\label{sec:analysis}
% Analysis

\subsection{Staleness Dynamics Visualization}

Figure~\ref{fig:staleness} visualizes the adaptive staleness values for each modality during training on AVE. In early epochs, the audio modality (which learns faster) receives higher staleness, reducing its update frequency. As training progresses and modalities converge, staleness values equalize.

% \begin{figure}[t]
%     \centering
%     \includegraphics[width=0.8\linewidth]{figures/staleness_dynamics.pdf}
%     \caption{Staleness dynamics during training on AVE.}
%     \label{fig:staleness}
% \end{figure}

\subsection{Prime Learning Window Extension}

A key claim of \method{} is that adaptive staleness prolongs the Prime Learning Window. To verify this, we track the epoch at which the performance gap between modalities exceeds a threshold (indicating one modality has ``won''). Figure~\ref{fig:prime_window} shows that \method{} extends this window compared to baselines.

% \begin{figure}[t]
%     \centering
%     \includegraphics[width=0.8\linewidth]{figures/prime_window.pdf}
%     \caption{Prime Learning Window duration across methods.}
%     \label{fig:prime_window}
% \end{figure}

\subsection{Hyperparameter Sensitivity}

We analyze sensitivity to key hyperparameters on CREMA-D.

\textbf{Maximum Staleness $\tau_{\max}$.} Table~\ref{tab:sens_tau} shows performance across $\tau_{\max} \in \{3, 5, 7, 10\}$. Performance is stable across a wide range, with slight degradation at very high staleness.

\textbf{Signal Balance $\beta$.} Both gradient magnitude ($\beta=1$) and loss descent ($\beta=0$) alone provide improvements, but combining them ($\beta=0.5$) achieves best results.

\begin{table}[t]
\centering
\caption{Sensitivity to $\tau_{\max}$ on CREMA-D.}
\label{tab:sens_tau}
\begin{tabular}{ccccc}
\toprule
$\tau_{\max}$ & 3 & 5 & 7 & 10 \\
\midrule
Acc & XX.XX & XX.XX & XX.XX & XX.XX \\
\bottomrule
\end{tabular}
\end{table}

\subsection{Computational Cost}

\method{} introduces minimal overhead compared to standard training. The staleness computation requires storing gradient norms and loss values (negligible memory). Update skipping actually \textit{reduces} computation—when a modality doesn't update, we skip its backward pass for that encoder.

\begin{table}[t]
\centering
\caption{Training time comparison on CREMA-D (seconds per epoch).}
\label{tab:time}
\begin{tabular}{lccc}
\toprule
Method & Time/Epoch & Relative \\
\midrule
Concatenation & XX.X & 1.00$\times$ \\
ARL & XX.X & 1.XX$\times$ \\
\method{} & XX.X & 0.XX$\times$ \\
\bottomrule
\end{tabular}
\end{table}


%------------------------------------------------------------------------------
\section{Conclusion}
\label{sec:conclusion}
% Conclusion

We presented \methodfull{} (\method{}), a novel approach to balanced multimodal learning through adaptive asynchronous updates. Unlike existing methods that modulate gradient magnitudes while maintaining synchronous updates, \method{} introduces true asynchrony where each modality has an independent update frequency determined by its learning dynamics.

By monitoring gradient magnitude ratios and loss descent rates, \method{} identifies fast-learning modalities and increases their staleness (reducing update frequency), while accelerating slow-learning modalities. This mechanism directly prolongs the Prime Learning Window, preventing dominant modalities from suppressing weaker ones during critical early training phases.

Our theoretical analysis connects \method{} to bounded-staleness SGD convergence guarantees from distributed optimization, adapting these results to modality-specific adaptive bounds. Experiments on CREMA-D, AVE, and Kinetics-Sounds demonstrate state-of-the-art performance, with particularly strong gains when modality imbalance is severe.

\textbf{Limitations and Future Work.} The current staleness computation relies on moving averages of gradient norms and losses, which may be noisy in early training. Future work could explore more sophisticated learning state estimation, such as using validation performance or learned probes. Additionally, extending \method{} to more complex fusion architectures (e.g., cross-modal attention) and more than two modalities would broaden its applicability.


%------------------------------------------------------------------------------
% References
\bibliographystyle{plainnat}
\bibliography{references}

%------------------------------------------------------------------------------
% Appendix
\appendix
\section{Theoretical Analysis}
\label{app:theory}
% Appendix: Theoretical Analysis

\subsection{Convergence of Bounded-Staleness SGD}

We first review convergence results for bounded-staleness SGD, then adapt them to \method{}.

\begin{theorem}[Bounded-Staleness SGD Convergence~\citep{}]
Consider minimizing $f(\theta)$ with bounded-staleness SGD where staleness $\tau \leq \tau_{\max}$. Under standard assumptions (L-smoothness, bounded variance), with learning rate $\eta = O(1/\sqrt{T\tau_{\max}})$:
\begin{equation}
    \frac{1}{T}\sum_{t=1}^{T} \mathbb{E}\|\nabla f(\theta_t)\|^2 \leq O\left(\frac{1}{\sqrt{T}} + \frac{\tau_{\max}}{\sqrt{T}}\right)
\end{equation}
\end{theorem}

\subsection{Extension to \method{}}

In \method{}, each modality $m_i$ has its own staleness bound $\tau_i \leq \tau_{\max}$. The key insight is that the multimodal loss decomposes as:
\begin{equation}
    \mathcal{L}(\theta_1, \ldots, \theta_N, \theta_f) = \mathcal{L}_{\text{fusion}} + \sum_{i=1}^{N} \mathcal{L}_i(\theta_i)
\end{equation}

Each term $\mathcal{L}_i$ depends only on $\theta_i$, allowing independent staleness bounds. The fusion loss $\mathcal{L}_{\text{fusion}}$ depends on all parameters but is always updated (staleness 0 for $\theta_f$).

\begin{proposition}[\method{} Convergence]
Under the same assumptions as Theorem 1, \method{} with adaptive staleness $\tau_i(t) \leq \tau_{\max}$ for all $i, t$ converges at rate:
\begin{equation}
    \frac{1}{T}\sum_{t=1}^{T} \mathbb{E}\|\nabla \mathcal{L}(\theta_t)\|^2 \leq O\left(\frac{1}{\sqrt{T}} + \frac{N \cdot \tau_{\max}}{\sqrt{T}}\right)
\end{equation}
\end{proposition}

The factor of $N$ (number of modalities) is unavoidable when each modality can have independent staleness. In practice, $N$ is small (2-5 modalities), so this does not significantly impact convergence.

\subsection{Gradient Compensation Analysis}

The gradient scaling factor $(1 + \lambda\tau_i)$ compensates for accumulated gradients during staleness. Under the approximation that gradients change slowly:
\begin{equation}
    \sum_{s=t-\tau_i}^{t-1} \nabla_{\theta_i}\mathcal{L}_i(\theta_s) \approx \tau_i \cdot \nabla_{\theta_i}\mathcal{L}_i(\theta_t)
\end{equation}

Thus scaling by $(1 + \lambda\tau_i)$ with $\lambda \approx 1$ recovers the accumulated gradient signal. In practice, we use smaller $\lambda$ for stability.


\section{Additional Experiments}
\label{app:experiments}
% Appendix: Additional Experiments

\subsection{Extended Hyperparameter Analysis}

\subsubsection{Baseline Staleness $\tau_{\text{base}}$}

Table~\ref{tab:app_tau_base} shows sensitivity to $\tau_{\text{base}}$ on CREMA-D.

\begin{table}[h]
\centering
\caption{Sensitivity to $\tau_{\text{base}}$ on CREMA-D.}
\label{tab:app_tau_base}
\begin{tabular}{cccccc}
\toprule
$\tau_{\text{base}}$ & 1 & 2 & 3 & 4 & 5 \\
\midrule
Acc & XX.XX & XX.XX & XX.XX & XX.XX & XX.XX \\
F1 & XX.XX & XX.XX & XX.XX & XX.XX & XX.XX \\
\bottomrule
\end{tabular}
\end{table}

\subsubsection{Gradient Compensation $\lambda$}

\begin{table}[h]
\centering
\caption{Sensitivity to $\lambda$ on CREMA-D.}
\label{tab:app_lambda}
\begin{tabular}{cccccc}
\toprule
$\lambda$ & 0 & 0.05 & 0.1 & 0.2 & 0.5 \\
\midrule
Acc & XX.XX & XX.XX & XX.XX & XX.XX & XX.XX \\
F1 & XX.XX & XX.XX & XX.XX & XX.XX & XX.XX \\
\bottomrule
\end{tabular}
\end{table}

\subsubsection{Loss Window Size $k$}

\begin{table}[h]
\centering
\caption{Sensitivity to loss window size $k$ on CREMA-D.}
\label{tab:app_k}
\begin{tabular}{cccccc}
\toprule
$k$ & 5 & 10 & 15 & 20 & 30 \\
\midrule
Acc & XX.XX & XX.XX & XX.XX & XX.XX & XX.XX \\
F1 & XX.XX & XX.XX & XX.XX & XX.XX & XX.XX \\
\bottomrule
\end{tabular}
\end{table}

\subsection{Per-Modality Performance}

Table~\ref{tab:app_modality} shows performance of individual modality encoders within the multimodal model, comparing baseline concatenation to \method{}.

\begin{table}[h]
\centering
\caption{Per-modality encoder performance on CREMA-D.}
\label{tab:app_modality}
\begin{tabular}{lcccc}
\toprule
\multirow{2}{*}{Method} & \multicolumn{2}{c}{Audio Encoder} & \multicolumn{2}{c}{Visual Encoder} \\
\cmidrule(lr){2-3} \cmidrule(lr){4-5}
& Acc & F1 & Acc & F1 \\
\midrule
Concatenation & XX.XX & XX.XX & XX.XX & XX.XX \\
\method{} & XX.XX & XX.XX & XX.XX & XX.XX \\
\bottomrule
\end{tabular}
\end{table}

\subsection{Different Backbone Architectures}

\begin{table}[h]
\centering
\caption{Performance with different backbones on CREMA-D.}
\label{tab:app_backbone}
\begin{tabular}{llcc}
\toprule
Backbone & Method & Acc & F1 \\
\midrule
\multirow{2}{*}{ResNet18} & Baseline & 58.83 & 59.43 \\
& \method{} & XX.XX & XX.XX \\
\midrule
\multirow{2}{*}{ResNet50} & Baseline & XX.XX & XX.XX \\
& \method{} & XX.XX & XX.XX \\
\midrule
\multirow{2}{*}{Swin-Base} & Baseline & 77.66 & -- \\
& \method{} & XX.XX & XX.XX \\
\bottomrule
\end{tabular}
\end{table}

\subsection{Three-Modality Experiment (MOSI)}

To demonstrate \method{}'s scalability beyond two modalities, we evaluate on MOSI which contains audio, visual, and text modalities.

\begin{table}[h]
\centering
\caption{Results on MOSI (3 modalities).}
\label{tab:app_mosi}
\begin{tabular}{lcc}
\toprule
Method & Acc & F1 \\
\midrule
Concatenation & 76.92 & 75.68 \\
ARL & 79.94 & 78.84 \\
\method{} & XX.XX & XX.XX \\
\bottomrule
\end{tabular}
\end{table}


\end{document}
